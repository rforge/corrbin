\newcommand{\NWtarget}[2]{#2}
\newcommand{\NWlink}[2]{#2}
\newcommand{\NWtxtMacroDefBy}{Fragment defined by}
\newcommand{\NWtxtMacroRefIn}{Fragment referenced in}
\newcommand{\NWtxtMacroNoRef}{Fragment never referenced}
\newcommand{\NWtxtDefBy}{Defined by}
\newcommand{\NWtxtRefIn}{Referenced in}
\newcommand{\NWtxtNoRef}{Not referenced}
\newcommand{\NWtxtFileDefBy}{File defined by}
\newcommand{\NWtxtIdentsUsed}{Uses:}
\newcommand{\NWtxtIdentsNotUsed}{Never used}
\newcommand{\NWtxtIdentsDefed}{Defines:}
\newcommand{\NWsep}{${\diamond}$}
\newcommand{\NWnotglobal}{(not defined globally)}
\newcommand{\NWuseHyperlinks}{}
\documentclass[reqno]{amsart}
\usepackage[margin=1in]{geometry}
\usepackage[colorlinks=true,linkcolor=blue]{hyperref}
\renewcommand{\NWtarget}[2]{\hypertarget{#1}{#2}}
\renewcommand{\NWlink}[2]{\hyperlink{#1}{#2}}

\newcommand{\pvec}{\mathbf{p}}
\def\T{{ \mathrm{\scriptscriptstyle T} }}
\newcommand{\setJ}{\mathcal{J}}
\newtheorem{theorem}{Theorem}


\title{Test for trend with a multinomial outcome}
\author{Aniko Szabo}
\date{\today}

\begin{document}
\maketitle


\section{Introduction}
-
Consider a study in which a multinomial outcome with $K$ possible unordered values is measured in subjects belonging to one of $G$ ordered groups.
The size of each group, $n_{i\cdot}$, is defined by the study design, and will be treated as fixed.
Let $\pvec_i=(p_{i1},\ldots,p_{iK})^\T$ denote the probabilities of the multinomial outcomes in the $i$th group. The hypothesis of interest is to
evaluate the homogeneity of these probabilities across the groups with a targeted alternative of a trend in at least one of the categories.
Formally, we consider testing $H_0 = \bigcap_{j=1}^K H_{0j}$ versus $H_1 = \bigcup_{j=1}^K H_{1j}$, where
\begin{equation}
\begin{aligned}
  H_{0j}& :  p_{1j}=\cdots = p_{Gj} \\
  H_{1j}& : p_{1j} \leq \cdots \leq p_{Gj} \text{ or }  p_{1j} \geq \cdots \geq p_{Gj} \text{ with at least one inequality}
\end{aligned}
\end{equation}

The test is based on the following result:
\begin{theorem}\label{Th:partial}
Let $\setJ \subset \{1,\ldots,K\}$, then under $H_{0\setJ}=\bigcap_{j\in\setJ}H_{0j}$ as $N\rightarrow\infty$
\begin{equation}
  W_\setJ =  \sum_{j\in\setJ} (1-p_{\cdot j})T^2_j + \big(\sum_{j\in\setJ} p_{\cdot j}\big) T^2_\setJ \xrightarrow{d} \chi^2_{d},
\end{equation}
where $d = \min(|\setJ|, K-1)$,  $T_\setJ = [\sum_{i=1}^G \sum_{j\in\setJ}n_{ij}(c_i-\bar{c})] \big/ \sqrt{p_{\cdot \setJ}(1-p_{\cdot \setJ})s^2}$ denotes the Cochran-Armitage trend test statistic for testing for marginal trend in $p_{i\setJ}=\sum_{j\in\setJ}p_{ij}$, $i=1,\ldots, G$.
\end{theorem}

\section{Implementing the overall test}

The main \texttt{multiCA.test} function is a generic, with methods for a matrix and formula input.

\begin{flushleft} \small
\begin{minipage}{\linewidth}\label{scrap1}\raggedright\small
\NWtarget{nuweb2a}{} \verb@"../R/aaa-generics.R"@\nobreak\ {\footnotesize {2a}}$\equiv$
\vspace{-1ex}
\begin{list}{}{} \item
\mbox{}\verb@@\\
\mbox{}\verb@#'Multinomial Cochran-Armitage trend test@\\
\mbox{}\verb@#'@\\
\mbox{}\verb@#'The \code{multiCA.test} performs a multinomial generalization of the @\\
\mbox{}\verb@#' Cochran-Armitage trend test.@\\
\mbox{}\verb@#'@\\
\mbox{}\verb@#'@\\
\mbox{}\verb@#'@{\tt @}\verb@export@\\
\mbox{}\verb@#'@{\tt @}\verb@param x a two-dimensional matrix or a formula@\\
\mbox{}\verb@#'@{\tt @}\verb@param \dots other arguments @\\
\mbox{}\verb@#'@{\tt @}\verb@return an object of class "htest" with the results of the test@\\
\mbox{}\verb@#'@{\tt @}\verb@author Aniko Szabo@\\
\mbox{}\verb@#'@{\tt @}\verb@references Szabo, A. (2016) Test for trend with a multinomial outcome.  @\\
\mbox{}\verb@#'@{\tt @}\verb@keywords nonparametric @\\
\mbox{}\verb@#'@{\tt @}\verb@examples@\\
\mbox{}\verb@#'@\\
\mbox{}\verb@#'data(stroke)@\\
\mbox{}\verb@#'## using formula interface@\\
\mbox{}\verb@#'multiCA.test(Type ~ Year, weights=Freq, data=stroke)@\\
\mbox{}\verb@#'@\\
\mbox{}\verb@#'## using matrix interface and testing only the first 3 outcomes@\\
\mbox{}\verb@#'strk.mat <- xtabs(Freq ~ Type + Year, data=stroke)@\\
\mbox{}\verb@#'multiCA.test(strk.mat, outcomes=1:3)@\\
\mbox{}\verb@#'@\\
\mbox{}\verb@#'@{\tt @}\verb@name multiCA.test@\\
\mbox{}\verb@@\\
\mbox{}\verb@multiCA.test <- function(x,...) UseMethod("multiCA.test")@\\
\mbox{}\verb@ @\\
\mbox{}\verb@@{\NWsep}
\end{list}
\vspace{-1.5ex}
\footnotesize
\begin{list}{}{\setlength{\itemsep}{-\parsep}\setlength{\itemindent}{-\leftmargin}}

\item{}
\end{list}
\end{minipage}\vspace{4ex}
\end{flushleft}
The actual calculation of the test statitistic, overall and unadjusted individual p-values is encapsulated in an internal function that operates on a matrix. No error control is provided here.

\begin{flushleft} \small\label{scrap2}\raggedright\small
\NWtarget{nuweb2b}{} \verb@"../R/multiCA.R"@\nobreak\ {\footnotesize {2b}}$\equiv$
\vspace{-1ex}
\begin{list}{}{} \item
\mbox{}\verb@@\\
\mbox{}\verb@#'@{\tt @}\verb@keywords internal@\\
\mbox{}\verb@@\\
\mbox{}\verb@.multiCA.test <- function(x, scores, outcomes){@\\
\mbox{}\verb@  K <- nrow(x)@\\
\mbox{}\verb@  full <- length(outcomes) == K  #full test@\\
\mbox{}\verb@  @\\
\mbox{}\verb@  nidot <- apply(x, 2, sum)@\\
\mbox{}\verb@  n <- sum(nidot)@\\
\mbox{}\verb@  @\\
\mbox{}\verb@  cbar <- sum(nidot * scores)/n@\\
\mbox{}\verb@  @\\
\mbox{}\verb@  s2 <- sum(nidot * (scores - cbar)^2)@\\
\mbox{}\verb@  pdot <- prop.table(rowSums(x))[outcomes]@\\
\mbox{}\verb@  nonz <- (pdot > 0)@\\
\mbox{}\verb@  @\\
\mbox{}\verb@  if (!any(nonz)) return(1)@\\
\mbox{}\verb@  @\\
\mbox{}\verb@  X <- x[outcomes, ,drop=FALSE] %*% (scores - cbar)@\\
\mbox{}\verb@@\\
\mbox{}\verb@  #individual tests@\\
\mbox{}\verb@  CAT <- X[nonz]^2 / (pdot[nonz] * (1-pdot[nonz])) / s2 @\\
\mbox{}\verb@  CAT.p.value <- pchisq(CAT, df=1, lower.tail=FALSE)@\\
\mbox{}\verb@  @\\
\mbox{}\verb@  #overall test@\\
\mbox{}\verb@  if (full || sum(pdot) >= 1){@\\
\mbox{}\verb@    Tt <- ( sum(X[nonz]^2 / pdot[nonz])) / s2@\\
\mbox{}\verb@  } else {@\\
\mbox{}\verb@    Tt <- (sum(X)^2 / (1-sum(pdot)) + sum(X[nonz]^2 / pdot[nonz])) / s2@\\
\mbox{}\verb@  }@\\
\mbox{}\verb@@\\
\mbox{}\verb@  df <- length(outcomes) - full@\\
\mbox{}\verb@  p.value <- pchisq(Tt, df=df, lower.tail=FALSE)@\\
\mbox{}\verb@@\\
\mbox{}\verb@  res <- list(statistic = Tt, parameter = df, p.value = p.value, @\\
\mbox{}\verb@              indiv.statistics = CAT, indiv.p.value = CAT.p.value)@\\
\mbox{}\verb@  return(res)@\\
\mbox{}\verb@}@\\
\mbox{}\verb@@{\NWsep}
\end{list}
\vspace{-1.5ex}
\footnotesize
\begin{list}{}{\setlength{\itemsep}{-\parsep}\setlength{\itemindent}{-\leftmargin}}
\item \NWtxtFileDefBy\ \NWlink{nuweb2b}{2b}\NWlink{nuweb?}{, ?}\NWlink{nuweb?}{, ?}.
\item \NWtxtIdentsDefed\nobreak\  \verb@.multiCA.test@\nobreak\ \NWlink{nuweb?}{?}.
\item{}
\end{list}
\vspace{4ex}
\end{flushleft}
The default method uses a two-dimensional contingency matrix with the outcomes as rows and ordered groups as columns.
\begin{flushleft} \small\label{scrap3}\raggedright\small
\NWtarget{nuweb?}{} \verb@"../R/multiCA.R"@\nobreak\ {\footnotesize {?}}$\equiv$
\vspace{-1ex}
\begin{list}{}{} \item
\mbox{}\verb@@\\
\mbox{}\verb@#'@{\tt @}\verb@rdname multiCA.test@\\
\mbox{}\verb@#'@{\tt @}\verb@method multiCA.test default@\\
\mbox{}\verb@#'@{\tt @}\verb@param scores numeric vector of the same length as the number of ordered groups. Defaults to linearly increasing values@\\
\mbox{}\verb@#'@{\tt @}\verb@param outcomes integer or character vector defining the set of outcomes (by row index or row name) over which the trend should be tested. Defaults to all outcomes.@\\
\mbox{}\verb@#'@{\tt @}\verb@export@\\
\mbox{}\verb@@\\
\mbox{}\verb@multiCA.test.default <- function(x, scores=1:ncol(x), outcomes=1:nrow(x),@\\
\mbox{}\verb@...){@\\
\mbox{}\verb@  if (!is.matrix(x)) {@\\
\mbox{}\verb@    cat(str(x))@\\
\mbox{}\verb@    stop("x should be a two-dimensional matrix")@\\
\mbox{}\verb@}@\\
\mbox{}\verb@  if (length(scores) != ncol(x)) stop("The length of the score vector should equal the number of columns of x")@\\
\mbox{}\verb@@\\
\mbox{}\verb@  testres <- .multiCA.test(x=x, scores=scores, outcomes=outcomes)@\\
\mbox{}\verb@ @\\
\mbox{}\verb@  Tt <- c(W = testres$statistic)@\\
\mbox{}\verb@  df <- c(df = testres$parameter)@\\
\mbox{}\verb@@\\
\mbox{}\verb@  p.value <- testres$p.value@\\
\mbox{}\verb@  null.value <- 0@\\
\mbox{}\verb@  names(null.value) <- sprintf("slope for outcomes %s", deparse(substitute(outcomes)))@\\
\mbox{}\verb@@\\
\mbox{}\verb@  res <- list(statistic = Tt, parameter = df, p.value = p.value, @\\
\mbox{}\verb@              method="Multinomial Cochran-Armitage trend test",@\\
\mbox{}\verb@              alternative="two.sided",@\\
\mbox{}\verb@              null.value=null.value,@\\
\mbox{}\verb@              data.name = deparse(substitute(x)))@\\
\mbox{}\verb@  class(res) <- "htest"@\\
\mbox{}\verb@  return(res)  @\\
\mbox{}\verb@}@\\
\mbox{}\verb@@{\NWsep}
\end{list}
\vspace{-1.5ex}
\footnotesize
\begin{list}{}{\setlength{\itemsep}{-\parsep}\setlength{\itemindent}{-\leftmargin}}
\item \NWtxtFileDefBy\ \NWlink{nuweb2b}{2b}\NWlink{nuweb?}{, ?}\NWlink{nuweb?}{, ?}.
\item \NWtxtIdentsDefed\nobreak\  \verb@multiCA.test.default@\nobreak\ \NWtxtIdentsNotUsed.\item \NWtxtIdentsUsed\nobreak\  \verb@.multiCA.test@\nobreak\ \NWlink{nuweb2b}{2b}.
\item{}
\end{list}
\vspace{4ex}
\end{flushleft}
The formula interface converts data into the appropriate contingency matrix for use
with the default method.

\begin{flushleft} \small\label{scrap4}\raggedright\small
\NWtarget{nuweb?}{} \verb@"../R/multiCA.R"@\nobreak\ {\footnotesize {?}}$\equiv$
\vspace{-1ex}
\begin{list}{}{} \item
\mbox{}\verb@@\\
\mbox{}\verb@#'@{\tt @}\verb@rdname multiCA.test@\\
\mbox{}\verb@#'@{\tt @}\verb@method multiCA.test formula@\\
\mbox{}\verb@#'@{\tt @}\verb@param formula a formula of the form \code{outcome ~ group} where \code{outcome} is a factor representing the cateogrical outcome and \code{group} is the grouping variableover which the trend is tested.@\\
\mbox{}\verb@#'@{\tt @}\verb@param data  an optional matrix or data frame containing the variables in the formula \code{formula}. By default the variables are taken from \code{environment(formula).}@\\
\mbox{}\verb@#'@{\tt @}\verb@param subset  an optional vector specifying a subset of observations to be used.@\\
\mbox{}\verb@#'@{\tt @}\verb@param na.action       a function which indicates what should happen when the data contain NAs. Defaults to getOption("na.action").@\\
\mbox{}\verb@#'@{\tt @}\verb@param weights an integer-valued variable representing the number of times each \code{outcome} - \code{group} combination was observed.@\\
\mbox{}\verb@#'@{\tt @}\verb@export@\\
\mbox{}\verb@@\\
\mbox{}\verb@multiCA.test.formula <- function(formula, data, subset, na.action,  weights, ...){@\\
\mbox{}\verb@    if (missing(formula) || (length(formula) != 3L) || (length(attr(terms(formula[-2L]), @\\
\mbox{}\verb@        "term.labels")) != 1L)) @\\
\mbox{}\verb@        stop("'formula' missing or incorrect")@\\
\mbox{}\verb@    m <- match.call(expand.dots = FALSE)@\\
\mbox{}\verb@    if (is.matrix(eval(m$data, parent.frame()))) @\\
\mbox{}\verb@        m$data <- as.data.frame(data)@\\
\mbox{}\verb@    m[[1L]] <- quote(stats::model.frame)@\\
\mbox{}\verb@    m$... <- NULL@\\
\mbox{}\verb@    mf <- eval(m, parent.frame())@\\
\mbox{}\verb@    responsevar <- attr(attr(mf, "terms"), "response")@\\
\mbox{}\verb@    response <- mf[[responsevar]]@\\
\mbox{}\verb@    weightvar <- which(names(mf)=="(weights)")@\\
\mbox{}\verb@    w <- if(length(weightvar) > 0)  mf[[weightvar]] else rep(1L, nrow(mf))@\\
\mbox{}\verb@    g <- factor(mf[,-c(responsevar, weightvar)])@\\
\mbox{}\verb@@\\
\mbox{}\verb@    tab <- xtabs(w ~ response + g)@\\
\mbox{}\verb@    multiCA.test(tab, ...)@\\
\mbox{}\verb@}@\\
\mbox{}\verb@@{\NWsep}
\end{list}
\vspace{-1.5ex}
\footnotesize
\begin{list}{}{\setlength{\itemsep}{-\parsep}\setlength{\itemindent}{-\leftmargin}}
\item \NWtxtFileDefBy\ \NWlink{nuweb2b}{2b}\NWlink{nuweb?}{, ?}\NWlink{nuweb?}{, ?}.
\item \NWtxtIdentsDefed\nobreak\  \verb@multiCA.test.formula@\nobreak\ \NWtxtIdentsNotUsed.
\item{}
\end{list}
\vspace{4ex}
\end{flushleft}
\section{Multiple testing adjusted inference for individual outcomes}

\subsection{Holm-Schaffer approach}


\section{Files}


{\small\begin{list}{}{\setlength{\itemsep}{-\parsep}\setlength{\itemindent}{-\leftmargin}}
\item \verb@"../R/aaa-generics.R"@ {\footnotesize {\NWtxtDefBy} \NWlink{nuweb2a}{2a}.}
\item \verb@"../R/multiCA.R"@ {\footnotesize {\NWtxtDefBy} \NWlink{nuweb2b}{2b}\NWlink{nuweb?}{, ?}\NWlink{nuweb?}{, ?}.
}
\end{list}}

\section{Macros}

None.


\section{Identifiers}


{\small\begin{list}{}{\setlength{\itemsep}{-\parsep}\setlength{\itemindent}{-\leftmargin}}
\item \verb@.multiCA.test@: \underline{\NWlink{nuweb2b}{2b}}\NWlink{nuweb?}{, ?}.
\item \verb@multiCA.test.default@: \underline{\NWlink{nuweb?}{?}}.
\item \verb@multiCA.test.formula@: \underline{\NWlink{nuweb?}{?}}.
\end{list}}

\end{document}
